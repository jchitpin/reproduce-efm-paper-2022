\documentclass[varwidth,border=0pt]{standalone}

% Tikz packages
\usepackage{tikz}
\usetikzlibrary{%
  patterns, plotmarks, backgrounds, shapes, arrows, calc, trees, positioning,
  chains, shapes.geometric, decorations.pathreplacing,
  decorations.pathmorphing, shapes.arrows, decorations.markings, quotes,
  arrows.meta, spy, fit, matrix, pgfplots.groupplots
}
\usepackage{pgfplotstable}
\usetikzlibrary{pgfplots.groupplots}

% General includegraphics and colour support
\usepackage{graphicx}
\usepackage{xcolor}

% Captions and subcaptions
\usepackage{caption}
\usepackage[labelformat=parens]{subcaption}

% String macros (for reading in tabular data)
\usepackage{xstring}

% PGFPlots
\pgfplotsset{%
  compat=newest,
  plot coordinates/math parser=false
}

%% Define node types
\tikzstyle{lnode} = [%
  circle,
  draw=black,
  minimum height=0.65cm,
  align=center,
  fill=none,
  text centered,
  inner sep=0.5pt,
  font=\tiny
]%

\renewcommand\thesubfigure{\alph{subfigure})}

\begin{document}

  \begin{figure}
    \begin{subfigure}[t]{0.33\textwidth}
      \caption{}
      \centering
      \scalebox{0.48}{\usetikzlibrary{pgfplots.groupplots}

% override style for non-boxed plots which is the case for both sub-plots
\pgfplotsset{every non boxed x axis/.style={}}

\pgfplotsset{%
  mystyle/.style={%
    scatter, only marks, scatter src=explicit symbolic, fill opacity=1.0,
    mark size=2, scatter/classes={%
      Markov={mark=o, fill=white},
      L2_norm={mark=square*,fill=white!0!black},
      qp_max_spa={mark=square*,fill=white!20!black},
      lp_max_spa={mark=square*,fill=white!40!black},
      lp_min_spa={mark=square*,fill=white!60!black},
      milp={mark=square*,fill=white!80!black}
    }
  }
}
\pgfplotstableread[col sep=tab]{../../data/barplot-zero-weight-freq-wt.csv}\mydata

\begin{tikzpicture}
  \begin{groupplot}[
    group style={%
      group name=wildtype,
      group size=1 by 2,
      xticklabels at=edge bottom,
      vertical sep=0pt
    },
    width=8.5cm,
    xmin=0,
    xmax=56
  ]

    \nextgroupplot[%
      xmajorticks=false,
      axis x line=top,
      ymin=-10,
      ymax=0,
      ytick={-9, -6, -3, 0},
      ytick pos=left,
      %axis y discontinuity=parallel,
      ylabel = {EFM weight ($\log_{10}$)},
      title style={yshift=-5.5ex},
      ylabel shift={-6 pt},
      title={\textbf{Wildtype}},
      legend style={at={(1.001,0.2345)},anchor=east, font=\small},
      legend cell align={left}
      %height=4.5cm
    ]
      \addplot[mystyle] table [x index=0, y index=1, meta index=2, col sep=tab, skip first n=0] {../../data/scatterplot-total-efm-weights-wt-log.csv};
      \legend{Markov, Min L2 (2), Max qSPA (2), Max lSPA (8), Min lSPA (8), Min milAP (1)};

    \nextgroupplot[%
      %yshift=-1cm,
      ybar stacked,
      bar width=3.5pt,
      ymin=0,
      ymax=1,
      xtick pos=left,
      ytick pos=left,
      xmin=-0.5,
      xtick={0, 10, 20, 30, 40, 50},
      ytick={0, 1.0},
      height=3.0cm,
      xlabel={EFM indices sorted by Markov solution},
      ylabel={Zero weight (\%)},
      ylabel style={text width=2.5cm, text centered},
      ylabel shift={2 pt},
      set layers,
      extra y ticks=0.32,
      extra y tick labels={{}},
      %major tick length=0,
      extra y tick style={%
        ymajorgrids=true,
        grid style={%
          black,
          semithick,
          dashed,
          /pgfplots/on layer=axis foreground,
        }
      }
    ]

      \addplot [fill=white!00!black] table [y=S1, meta=Label, x expr=\coordindex]{\mydata};
      \addplot [fill=white!20!black] table [y=S2, meta=Label, x expr=\coordindex]{\mydata};
      \addplot [fill=white!40!black] table [y=S3, meta=Label, x expr=\coordindex]{\mydata};
      \addplot [fill=white!60!black] table [y=S4, meta=Label, x expr=\coordindex]{\mydata};
      \addplot [fill=white!80!black] table [y=S5, meta=Label, x expr=\coordindex]{\mydata};

  \end{groupplot}

  \pgfresetboundingbox\path (wildtype c1r1.outer north east) rectangle (wildtype c1r2.outer south west); % adjust to fit
\end{tikzpicture}

}
    \end{subfigure}\hfill%
    \begin{subfigure}[t]{0.33\textwidth}
      \caption{}
      \centering
      \scalebox{0.48}{\usetikzlibrary{pgfplots.groupplots}

% override style for non-boxed plots which is the case for both sub-plots
\pgfplotsset{every non boxed x axis/.style={}}
\pgfplotsset{%
  mystyle/.style={%
    scatter, only marks, scatter src=explicit symbolic, draw opacity=1,
    mark size=2, scatter/classes={%
      Markov={mark=o, fill=white},
      L2_norm={mark=square*,fill=white!00!black},
      qp_max_spa={mark=square*,fill=white!20!black},
      lp_max_spa={mark=square*,fill=white!40!black},
      lp_min_spa={mark=square*,fill=white!60!black},
      milp={mark=square*,fill=white!80!black}
    }
  }
}
\pgfplotstableread[col sep=tab]{../../data/barplot-zero-weight-freq-ad.csv}\mydata

\begin{tikzpicture}
  \begin{groupplot}[
    group style={%
      group name=disease,
      group size=1 by 2,
      xticklabels at=edge bottom,
      vertical sep=0pt
    },
    width=8.5cm,
    xmin=0,
    xmax=56
  ]

    \nextgroupplot[%
      xmajorticks=false,
      axis x line=top,
      ymin=-10,
      ymax=0,
      ytick={-9, -6, -3, 0},
      ytick pos=left,
      %axis y discontinuity=parallel,
      ylabel = {EFM weight ($\log_{10}$)},
      ylabel shift={-6 pt},
      title style={yshift=-5.5ex},
      title={\textbf{Alzheimer's disease}},
      legend style={at={(1.001,0.2345)},anchor=east, font=\small},
      legend cell align={left}
      %height=4.5cm
    ]
      \addplot[mystyle] table [x index=0, y index=1, meta index=2, col sep=tab, skip first n=0] {../../data/scatterplot-total-efm-weights-ad-log.csv};
      %\legend{Markov, Min L2 (2), Max qSPA (2), Max lSPA (8), Min lSPA (8), Min milAP (1)};

    \nextgroupplot[%
      %yshift=-1cm,
      ybar stacked,
      bar width=3.5pt,
      ymin=0,
      ymax=1,
      xmin=-0.5,
      xtick pos=left,
      ytick pos=left,
      xtick={0, 10, 20, 30, 40, 50},
      ytick={0, 1.0},
      height=3.0cm,
      xlabel={EFM indices sorted by Markov solution},
      ylabel={Zero weight (\%)},
      ylabel style={text width=2.5cm, text centered},
      ylabel shift={2 pt},
      set layers,
      extra y ticks=0.35,
      extra y tick labels={{}},
      %major tick length=0,
      extra y tick style={%
        ymajorgrids=true,
        grid style={%
          black,
          semithick,
          dashed,
          /pgfplots/on layer=axis foreground,
        }
      }
    ]

      \addplot [fill=white!00!black] table [y=S1, meta=Label, x expr=\coordindex]{\mydata};
      \addplot [fill=white!20!black] table [y=S2, meta=Label, x expr=\coordindex]{\mydata};
      \addplot [fill=white!40!black] table [y=S3, meta=Label, x expr=\coordindex]{\mydata};
      \addplot [fill=white!60!black] table [y=S4, meta=Label, x expr=\coordindex]{\mydata};
      \addplot [fill=white!80!black] table [y=S5, meta=Label, x expr=\coordindex]{\mydata};

  \end{groupplot}

  \pgfresetboundingbox\path (disease c1r1.outer north east) rectangle (disease c1r2.outer south west); % adjust to fit
\end{tikzpicture}

}
    \end{subfigure}
    \begin{subfigure}[t]{0.33\textwidth}
      \caption{}
      \centering
      \vspace*{0.11cm}
      \scalebox{0.48}{\pgfplotstableread[col sep=tab]{../../data/scatterplot-fold-change.csv}\mydata

\pgfplotsset{%
  mc/.style={only marks, fill opacity=1.0, mark size=2, mark=o, fill=white},
  sk/.style={only marks, fill opacity=1.0, mark size=2, mark=square*, fill=white!0!black},
  or/.style={only marks, fill opacity=1.0, mark size=2, mark=square*, fill=white!20!black},
  ru/.style={only marks, fill opacity=1.0, mark size=2, mark=square*, fill=white!40!black},
  re/.style={only marks, fill opacity=1.0, mark size=2, mark=square*, fill=white!60!black},
  no/.style={only marks, fill opacity=1.0, mark size=2, mark=square*, fill=white!80!black}
}

\begin{tikzpicture}

  \begin{groupplot}[%
    group style={%
      vertical sep=0pt,
      horizontal sep=0pt,
      group size=6 by 1,
      x descriptions at=edge bottom,
      y descriptions at=edge left,
    },
    xtick pos=left,
    ytick pos=left,
    %xmin=0,
    ymin=-17,
    ymax=12,
    width=1.11cm,
    height=7.175cm,
    scale only axis,
    title style={%
      at={(0.5,0)},
      anchor=north,
      yshift=-6pt
    },
    xmajorticks=false,
    ylabel shift={-6 pt},
    %ylabel style={font=\scriptsize},
    %ylabel={$\log_{2}(w_{ad} / w_{wt})$},
    %ylabel={$\log_{2}$(disease weight / wildtype weight)}
    ylabel={Fold change $(\log_2)$}
  ]

    \nextgroupplot[title=1]
      \draw[dashed,thin] (axis cs:\pgfkeysvalueof{/pgfplots/xmin},-2) -- (axis cs:\pgfkeysvalueof{/pgfplots/xmax},-2);
      \draw[dashed,thin] (axis cs:\pgfkeysvalueof{/pgfplots/xmin},+2) -- (axis cs:\pgfkeysvalueof{/pgfplots/xmax},+2);
      \addplot[mc, restrict x to domain=1:11] table [x index=0, y index=1, col sep=tab, skip first n=0] \mydata;
      \addplot[sk, restrict x to domain=1:11] table [x index=0, y index=2, col sep=tab, skip first n=0] \mydata;
      \addplot[or, restrict x to domain=1:11] table [x index=0, y index=3, col sep=tab, skip first n=0] \mydata;
      \addplot[ru, restrict x to domain=1:11] table [x index=0, y index=4, col sep=tab, skip first n=0] \mydata;
      \addplot[re, restrict x to domain=1:11] table [x index=0, y index=5, col sep=tab, skip first n=0] \mydata;
      \addplot[no, restrict x to domain=1:11] table [x index=0, y index=6, col sep=tab, skip first n=0] \mydata;
    \nextgroupplot[title=2, ytick style={/pgfplots/major tick length=0pt}]
      \draw[dashed,thin] (axis cs:\pgfkeysvalueof{/pgfplots/xmin},-2) -- (axis cs:\pgfkeysvalueof{/pgfplots/xmax},-2);
      \draw[dashed,thin] (axis cs:\pgfkeysvalueof{/pgfplots/xmin},+2) -- (axis cs:\pgfkeysvalueof{/pgfplots/xmax},+2);
      \addplot[mc, restrict x to domain=12:13] table [x index=0, y index=1, col sep=tab, skip first n=0] \mydata;
      \addplot[sk, restrict x to domain=12:13] table [x index=0, y index=2, col sep=tab, skip first n=0] \mydata;
      \addplot[or, restrict x to domain=12:13] table [x index=0, y index=3, col sep=tab, skip first n=0] \mydata;
      \addplot[ru, restrict x to domain=12:13] table [x index=0, y index=4, col sep=tab, skip first n=0] \mydata;
      \addplot[re, restrict x to domain=12:13] table [x index=0, y index=5, col sep=tab, skip first n=0] \mydata;
      \addplot[no, restrict x to domain=12:13] table [x index=0, y index=6, col sep=tab, skip first n=0] \mydata;
    \nextgroupplot[title=3, ytick style={/pgfplots/major tick length=0pt}]
      \draw[dashed,thin] (axis cs:\pgfkeysvalueof{/pgfplots/xmin},-2) -- (axis cs:\pgfkeysvalueof{/pgfplots/xmax},-2);
      \draw[dashed,thin] (axis cs:\pgfkeysvalueof{/pgfplots/xmin},+2) -- (axis cs:\pgfkeysvalueof{/pgfplots/xmax},+2);
      \addplot[mc, restrict x to domain=14:20] table [x index=0, y index=1, col sep=tab, skip first n=0] \mydata;
      \addplot[sk, restrict x to domain=14:20] table [x index=0, y index=2, col sep=tab, skip first n=0] \mydata;
      \addplot[or, restrict x to domain=14:20] table [x index=0, y index=3, col sep=tab, skip first n=0] \mydata;
      \addplot[ru, restrict x to domain=14:20] table [x index=0, y index=4, col sep=tab, skip first n=0] \mydata;
      \addplot[re, restrict x to domain=14:20] table [x index=0, y index=5, col sep=tab, skip first n=0] \mydata;
      \addplot[no, restrict x to domain=14:20] table [x index=0, y index=6, col sep=tab, skip first n=0] \mydata;
    \nextgroupplot[title=4, ytick style={/pgfplots/major tick length=0pt}]
      \draw[dashed,thin] (axis cs:\pgfkeysvalueof{/pgfplots/xmin},-2) -- (axis cs:\pgfkeysvalueof{/pgfplots/xmax},-2);
      \draw[dashed,thin] (axis cs:\pgfkeysvalueof{/pgfplots/xmin},+2) -- (axis cs:\pgfkeysvalueof{/pgfplots/xmax},+2);
      \addplot[mc, restrict x to domain=21:24] table [x index=0, y index=1, col sep=tab, skip first n=0] \mydata;
      \addplot[sk, restrict x to domain=21:24] table [x index=0, y index=2, col sep=tab, skip first n=0] \mydata;
      \addplot[or, restrict x to domain=21:24] table [x index=0, y index=3, col sep=tab, skip first n=0] \mydata;
      \addplot[ru, restrict x to domain=21:24] table [x index=0, y index=4, col sep=tab, skip first n=0] \mydata;
      \addplot[re, restrict x to domain=21:24] table [x index=0, y index=5, col sep=tab, skip first n=0] \mydata;
      \addplot[no, restrict x to domain=21:24] table [x index=0, y index=6, col sep=tab, skip first n=0] \mydata;
    \nextgroupplot[title=5, ytick style={/pgfplots/major tick length=0pt}]
      \draw[dashed,thin] (axis cs:\pgfkeysvalueof{/pgfplots/xmin},-2) -- (axis cs:\pgfkeysvalueof{/pgfplots/xmax},-2);
      \draw[dashed,thin] (axis cs:\pgfkeysvalueof{/pgfplots/xmin},+2) -- (axis cs:\pgfkeysvalueof{/pgfplots/xmax},+2);
      \addplot[mc, restrict x to domain=25:47] table [x index=0, y index=1, col sep=tab, skip first n=0] \mydata;
      \addplot[sk, restrict x to domain=25:47] table [x index=0, y index=2, col sep=tab, skip first n=0] \mydata;
      \addplot[or, restrict x to domain=25:47] table [x index=0, y index=3, col sep=tab, skip first n=0] \mydata;
      \addplot[ru, restrict x to domain=25:47] table [x index=0, y index=4, col sep=tab, skip first n=0] \mydata;
      \addplot[re, restrict x to domain=25:47] table [x index=0, y index=5, col sep=tab, skip first n=0] \mydata;
      \addplot[no, restrict x to domain=25:47] table [x index=0, y index=6, col sep=tab, skip first n=0] \mydata;
    \nextgroupplot[title=6, ytick style={/pgfplots/major tick length=0pt}]
      \draw[dashed,thin] (axis cs:\pgfkeysvalueof{/pgfplots/xmin},-2) -- (axis cs:\pgfkeysvalueof{/pgfplots/xmax},-2);
      \draw[dashed,thin] (axis cs:\pgfkeysvalueof{/pgfplots/xmin},+2) -- (axis cs:\pgfkeysvalueof{/pgfplots/xmax},+2);
      \addplot[mc, restrict x to domain=48:55] table [x index=0, y index=1, col sep=tab, skip first n=0] \mydata;
      \addplot[sk, restrict x to domain=48:55] table [x index=0, y index=2, col sep=tab, skip first n=0] \mydata;
      \addplot[or, restrict x to domain=48:55] table [x index=0, y index=3, col sep=tab, skip first n=0] \mydata;
      \addplot[ru, restrict x to domain=48:55] table [x index=0, y index=4, col sep=tab, skip first n=0] \mydata;
      \addplot[re, restrict x to domain=48:55] table [x index=0, y index=5, col sep=tab, skip first n=0] \mydata;
      \addplot[no, restrict x to domain=48:55] table [x index=0, y index=6, col sep=tab, skip first n=0] \mydata;
  \end{groupplot}
  % X/Y labels and title
  \node [white, below=4.8cm,anchor=south,rotate=0] at ($(group c6r1.south west)!0.5!(group c2r1.north west)$) {EFM indices sorted by compartments spanned};
  \node [below=4.615cm,anchor=south,rotate=0] at ($(group c6r1.south west)!0.5!(group c2r1.north west)$) {EFM indices sorted by compartments spanned};
  \node [above=2.900cm,anchor=south,rotate=0,fill=white] at ($(group c6r1.south west)!0.5!(group c2r1.north west)$) {\textbf{Disease vs wildtype}};

  % Attempted to plot the dominant EFMs explaining 95% fluxes but not readable
  %\node[anchor=north,font=\tiny] at (0.090,3.8) {$\uparrow$}; % 1
  %\node[anchor=north,font=\tiny] at (0.185,3.8) {$\uparrow$}; % 2
  %\node[anchor=north,font=\tiny] at (0.280,3.8) {$\uparrow$}; % 3
  %\node[anchor=north,font=\tiny] at (0.375,3.8) {$\uparrow$}; % 4
  %\node[anchor=north,font=\tiny] at (0.475,3.8) {$\uparrow$}; % 5
  %\node[anchor=north,font=\tiny] at (0.570,3.8) {$\uparrow$}; % 6
  %\node[anchor=north,font=\tiny] at (0.665,3.8) {$\uparrow$}; % 7
  %\node[anchor=north,font=\tiny] at (0.760,3.8) {$\uparrow$}; % 8
  %\node[anchor=north,font=\tiny] at (0.855,3.8) {$\uparrow$}; % 9
  %\node[anchor=north,font=\tiny] at (0.950,3.8) {$\uparrow$}; % 10
  %\node[anchor=north,font=\tiny] at (2.620,3.8) {$\uparrow$}; % 16
  %\node[anchor=north,font=\tiny] at (2.620,3.8) {$\uparrow$}; % 18
  %\node[anchor=north,font=\tiny] at (2.620,3.8) {$\uparrow$}; % 38

\end{tikzpicture}


}
    \end{subfigure}
  \end{figure}

\end{document}

